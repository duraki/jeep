% jeep.tex [jeep + penetration testing framework for vehicles]

% doclass 
\documentclass[twocolumn, a4paper, 10pt]{report}


% deps
\usepackage{lipsum}     % autogen lorem ipsum
\usepackage{titling}    % allow messing with titles
\usepackage{blindtext}  % manipualte loop
\usepackage{afterpage}  % page manipulation
\usepackage{color}      % lets give it a shine

% doc
\begin{document}

% -- REPORT COVER BEGIN --
% style
\pagecolor{black}
\color{white}

% report
\title{%
    jeep \\
    \large penetration testing framework for vehicle systems \\}
\author{Halis Duraki}
\date{\today \\
{\large https://duraki.github.io/jeep}}

\maketitle

% style
\pagecolor{white}
\color{black}

% report
\title{%
    jeep \\
    \large penetration testing framework for vehicle systems \\}
\author{Halis Duraki}
\date{\today}
\maketitle

% -- REPORT COVER END --

\begin{abstract}
We spend most of our time connected, available to the world, online. We
managed to overcome another obstacle when Internet of Things happened. 
We can positively say that in the near future, almost every device around us 
will communicate with each other. A new era is born quite fast, when news paper
started writing articles about "smart" and "AI" powered cars, running on electricity and
batteries.
Such cars, equiped with a special software that can recieve software updates over the 
network, [which] increase your vehicle HP, or for instance make your vehicle spend
less electricty while in use, already exists - like Tesla for example. But what other, older cars are running? How do
they work, if they aren't smart, nor mechanic? They must be connected somehow. 
By gathering information and analyzing CAN bus protocol, while constantly
disovering real life scenarios on vehicles, we developed a software framework which will help
security researchers to investigate, and debug vehicle systems. The framework is
exstensible in various ways, and offer several modules, some of them presented
in this paper. One can use our
framework in development or research purpose. We will finish the paper with demonstration
and a valid usage information.
\end{abstract}

\end{document}
